\documentclass{beamer}

\usepackage[utf8]{inputenc}
\usepackage[T1]{fontenc}
\usepackage{lmodern}
\usepackage[ngerman]{babel}

\usepackage{tikz}
\usetikzlibrary{positioning,shapes.geometric}
\usepackage{listings}

\usetheme{Luebeck}

\title{Meine erste Präsentation}
\author{Malte Schmitz}

\begin{document}
  \begin{frame}
    \maketitle
  \end{frame}

  \begin{frame}{Gliederung}
    \tableofcontents
  \end{frame}

  \section{Einführung}

  \begin{frame}{Funktionen von Beamer}
    Kompilieren wie jedes andere
    \LaTeX-Dokument auch.
  \end{frame}

  \section{Hauptteil}

  \subsection{Formel}

  \begin{frame}{Eine Formel}
    \[ a^2 + b^2 = c^2 \]
  \end{frame}

  \subsection{Sandhaufensatz}

  \begin{frame}{Sandhaufensatz}
    \begin{Satz}[Sandhaufensatz]
      Es gibt keine Sandhaufen.
    \end{Satz}

    \begin{Beweis}<2->
      \begin{enumerate}
        \item<3-> Ein Sandkorn ist kein Sandhaufen.
        \item<4-> Sandkörner werden durch Hinzufügen eines Sandkorns nicht zum Sandhaufen.
        \item Induktiv folgt die Aussage. \qedhere
      \end{enumerate}
      \end{Beweis}
    \onslide<5->
    Der Induktionsbeweis ist \alert<6>{falsch}!
  \end{frame}

  \subsection{Animation}

  \begin{frame}[fragile]{Animation}
    \tikzset{
      io/.style={
        trapezium,
        trapezium left angle=70,
        trapezium right angle=110,
        fill=magenta!10,
        draw=magenta
      },
      op/.style={
        rectangle,
        fill=orange!10,
        draw=orange
      },
      cn/.style={
        diamond,
        aspect=2,
        inner sep=2pt,
        fill=red!10,
        draw=red
      },
      node distance=5mm,
      thick
    }

    \lstset{
      morekeywords={function,var,while,return},
      basicstyle={\ttfamily},
      keywordstyle={\bfseries\color{orange}}
    }

    \begin{columns}
      \column{4cm}

      \begin{tikzpicture}
        \node[io] (in) {Eingabe $a,b$};
        \pause

        \node[op, below=of in] (div) {$r=a \mod b$};
        \path[->] (in) edge (div);
        \pause

        \node[op, below=of div] (set) {$a=b,\ b=r$};
        \path[->] (div) edge (set);
        \pause

        \node[cn, below=of set] (cond) {$b=0?$};
        \path[->] (set) edge (cond);
        \draw[->] (cond) -- node[below] {Nein} ++(1.5,0) |- (div);

        \pause
        \node[io, below=of cond] (out) {Ausgabe $a$};
        \path[->] (cond) edge node[right] {Ja} (out);

        \onslide<1->
      \end{tikzpicture}

      \column{5cm}
      \begin{lstlisting}[gobble=6]
        function ggt(a, b) {
          var r;
          while (b != 0) {
            r = a % b;
            a = b;
            b = r;
          }
          return a;
        }
      \end{lstlisting}
    \end{columns}

  \end{frame}

  \section{Schluss}

  \begin{frame}{Fazit}
    \begin{enumerate}
      \item Wir
      \item sind
      \item am
      \item Ende.
    \end{enumerate}
  \end{frame}
\end{document}