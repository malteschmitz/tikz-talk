%!TEX root = tikz.tex

\begin{frame}{Was ist \beamer?}
  \begin{itemize}
    \item \alert{Dokumentenklasse für \LaTeX} für die Erzeugung von Präsentationen.
      \only<article>{\newline(Diese Präsentation und das Skript wurden mit \beamer\ erzeugt.)}
    \item Keine eigene und \alert{keine graphische Anwendung}.
    \item \strut\beamer\ ist in vielen \TeX-Distributionen enthalten.\newline
      (\alert{Es kann direkt losgehen}.)
  \end{itemize}
\end{frame}

\begin{frame}{Workflow}
  \lstset{language={}}
  \begin{enumerate}
    \item Normales \LaTeX-Dokument erzeugen.\\
      Dabei einige spezielle \beamer-Kommandos verwenden.
    \item \LaTeX-Dokument mit \lstinline-pdflatex- kompilieren.
    \item Ergebnis überprüfen und \LaTeX-Dokument anpassen.
  \end{enumerate}
\end{frame}

\begin{frame}[fragile]{Folien}
  \begin{itemize}
    \item Ein \beamer-Dokument besteht aus mehreren Frames.
    \item Jeder Frame kann aus mehreren Slides bestehen.
    \item Die Umgebung \lstinline-frame- verarbeitet
      bis zu zwei Parameter in gescheiften Klammern \lstinline-{}-
    \item Der erste Parameter ist der Titel.
    \item Der zweite Parameter ist der Untertitel.
    \item Innerhalb der Umgebung \lstinline|frame| wird normaler \LaTeX-Code
      verwendet.
  \end{itemize}
\end{frame}

