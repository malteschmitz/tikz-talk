%!TEX root = tikz.tex

\begin{frame}{Zusammenfassung}
  \begin{enumerate}
    \item \TikZ-Zeichnungen bestehen aus \alert{Pfaden}, die über \alert{Koordinaten} definiert werden.
    \item Fast alle scheamtischen Zeichnungen sind ein \alert{Graph}, bestehen also aus \alert{Knoten} und \alert{Kanten} und
      werden auch als solche in \TikZ\ gezeichnet.
    \item Mit der Dokumentenklasse \lstinline-beamer- können \alert{sehr
      leicht Präsentationen erstellt} werden, wenn man mit \LaTeX\ etwas geübt ist.
    \item \alert{Overlay-Spezifikationen} werden in spitzen
      Klammern \lstinline-<- und \lstinline->- angegeben. Diese beeinflussen, in welchem
      \alert{Schritt der Animation} das Kommando ausgeführt wird.
    \item Bei Problemen und Fragen \alert{lies die Anleitung!}
  \end{enumerate}
\end{frame}

\begin{frame}{Zum Weiterlesen}
  \begin{mybib}
    \bibitem{Tantau4}
      Till Tantau.
      \newblock \emph{The \TikZ\ and \textsc{pgf} Packages},
      \newblock Manual for version 2.10,
      \newblock \alt<presentation>{\href{http://mirrors.ctan.org/graphics/pgf/base/doc/generic/pgf/pgfmanual.pdf}{\texttt{pgfmanual.pdf}}}{\url{http://mirrors.ctan.org/graphics/pgf/base/doc/generic/pgf/pgfmanual.pdf}}, Oktober 2010.

    \bibitem{Texample}
      Kjell Magne Fauske und Stefan Kottwitz.
      \newblock \emph{\TeX ample.net},
      \newblock ample resources for TeX users,
      \newblock \alt<presentation>{\href{http://www.texample.net/tikz/examples/}{\texttt{texample.net}}}{\url{http://www.texample.net/tikz/examples/}}.
  \end{mybib}
\end{frame}

\begin{frame}{Zum Weiterlesen (Beamer)}
  \begin{mybib}
    \bibitem{Tantau}
      Till Tantau, Joseph Wright und Vedran Mileti\'c.
      \newblock The \textsc{beamer} \textit{class}, User Guide.
      \newblock \alt<presentation>{\href{http://mirrors.ctan.org/macros/latex/contrib/beamer/doc/beameruserguide.pdf}{\texttt{beameruserguide.pdf}}}{\url{http://mirrors.ctan.org/macros/latex/contrib/beamer/doc/beameruserguide.pdf}}, Oktober 2013.

    \bibitem{Tantau}
      Till Tantau.
      \newblock \emph{Beamer: Strahlende Vorträge mit \LaTeX},
      \newblock Präsentieren und Dokumentieren -- Tools.
      \newblock Vorlesung vom 31. Oktober 2012.
  \end{mybib}
\end{frame}